%%%%%%%%%%%%%%%%%%%%%%%%%%%%%%%%%%%%%%%%%%%%%%%%%%%%%%%%%%%%%%%%%%%%%%%%%%%%%%%%%
%																				%
%                                ELEN4020.tex    			    				%
%						    	Emma Clark(1088496) 							%
%                               Jason Smit (709363)		        				%
%						      Isabel Tollman (728359)           				%
%                                                                   		    %
%                                                                               %
%																				%
%%%%%%%%%%%%%%%%%%%%%%%%%%%%%%%%%%%%%%%%%%%%%%%%%%%%%%%%%%%%%%%%%%%%%%%%%%%%%%%%%
\documentclass[hidelinks,10pt,onecolumn]{witseiepaper}
\pagenumbering{arabic} %set the style of numbering

\usepackage[none]{hyphenat}
\usepackage{flushend}
\usepackage{rotating} %
\usepackage{url} %for url's in the reference
\usepackage[square,comma,numbers,sort&compress]{natbib} %for referencing very important
\usepackage{balance} %balance the last page
\usepackage{amsmath} %math package
\usepackage{xr} %Reference appendix

\usepackage{hyperref}

% for putting code into the report : Looks really good
%%%%%%%%%%%%%%%%%%%%%%%%%%%%%%%%%%%%%%%%%%%%%%%%%%%%%%%%%%%%%%%%%%%%%%%%%%%%%%%%%
\usepackage{listings} 															%
\usepackage{color} 		%
																				%
\definecolor{dkgreen}{rgb}{0,0.6,0}												%
\definecolor{gray}{rgb}{0.5,0.5,0.5}											%
\definecolor{mauve}{rgb}{0.58,0,0.82}											%
																				%
\lstset{frame=tb,																%
  language=C,																%
  aboveskip=3mm,																%
  belowskip=3mm,																%
  showstringspaces=false,														%
  columns=flexible,																%
  basicstyle={\small\ttfamily},													%
  numbers=none,																	%
  numberstyle=\tiny\color{gray},												%			
  keywordstyle=\color{blue},													%
  commentstyle=\color{dkgreen},													%
  stringstyle=\color{mauve},													%
  breaklines=true,																%
  breakatwhitespace=true,														%
  tabsize=3																		%
}																				%
%%%%%%%%%%%%%%%%%%%%%%%%%%%%%%%%%%%%%%%%%%%%%%%%%%%%%%%%%%%%%%%%%%%%%%%%%%%%%%%%%


		\pdfinfo{
		/Title ()
		/Author (Jason R. Smit)
		/CreationDate (D:201711040830)
		/ModDate (D:201711052000)	
		/Subject (FINAL DRAFT)
		/Keywords ()
		}%needed to put information into a pdf discription file

%%%%%%%%%%%%%%%%%%%%%%%%%%%%%%%%%%%%%%%%%%%%%%%%%%%%%%%%%%%%%%%%%%%%%%%%%%%%%%%

%\usepackage{siunitx}
%%%%%%%%%%%%%%%%%%%%%%%%%%%%%%%%%%%%%%%%%%%%%%%%%%%%%%%%%%%%%%%%%%%%%%%%%%%%%%%
\begin{document}
%\begin{titlepage}

\title{\centering \textbf{{ELEN4020-LAB1}}}

\author{\centering {{Emma Clark \\ Jason Smit \\ Isabel Tollman}
{ \\ \normalfont \textit{School of  Electrical \& Information Engineering.} \\
\textit{University of the
Witwatersrand}\\
Private Bag 3, 2050, Johannesburg, South Africa \\}}}
%\thanks{School of  Electrical \& Information Engineering. University of the
%Witwatersrand, Private Bag 3, 2050, Johannesburg, South Africa}


%%%%%%%%%%%%%%%%%%%%%%%%%%%%%%%%%%%%%%%%%%%%%%%%%%%%%%%%%%%%%%%%%%%%%%%%%%%%%%%
% 
\abstract{}

\keywords{}


\maketitle
\thispagestyle{empty}\pagestyle{empty}
%\end{titlepage}

%%%%%%%%%%%%%%%%%%%%%%%%%%%%%%%%%%%%%%%%%%%%%%%%%%%%%%%%%%%%%%%%%%%%%%%%%%%%%%%
%

\section{INTRODUCTION} 
This report details the procedure for addition and multiplication of Tensors of rank 2D and 3D. For each section the procedure is outlined and pseudo code is presented. Each of the procedures is  using C++. 
\section{2D TENSOR ADD}
The addition of two 2-dimensional matrices is achieved using element-by-element addition, as shown in Equation \ref{2d-add-eqn}. Algorithm 1 displays the pseudo-code used to calculate the addition of two N$\times$N matrices.

\begin{align}
C_{ij} = A_{ij} + B_{ij} \label{2d-add-eqn}
\end{align}
\subsection{Code}
\begin{figure}[h!]
\includegraphics[width=\textwidth]{build/Algo1.png}
\end{figure}
\section{2D TENSOR MULTIPLY}
The multiplication of two 2-dimensional matrices is achieved by performing the dot product on the respective
rows and columns, as illustrated in Equation \ref{2d-mult-eqn}. Algorithm 2 shows the pseudo-code used to multiply two N$\times$N matrices.
\begin{align}
C_{ij} = \sum_{k} A_{ij} \times B_{ij} \label{2d-mult-eqn}
\end{align}

\subsection{Code}
\begin{figure}[!h]
\includegraphics[width=\textwidth]{build/Algo2.png}
\end{figure}
\section{3D TENSOR ADD}
The addition of two 3-dimensional arrays is achieved using element-by-element addition,similar to a 2D array. The difference is only the addition of the inclusion of the additional dimension. Algorithm 3 shows
the pseudo-code used to sum two N$\times$N$\times$N matrices.
\subsection{Code}
\begin{figure}[h!]
\includegraphics[width=\textwidth]{build/Algo3.png}
\end{figure}
\section{3D TENSOR MULTIPLY}
The multiplication of two 3-dimensional arrays makes use of 2-dimensional matrix multiplication as a basis. The $\text{i}^\text{th}$row-plane of array A and the $\text{j}^\text{th}$ column-plane of array B are multiplied using traditional 2-dimensional matrix multiplication shown in Algorithm \ref{2d-mult-eqn}. The result is the $\text{k}^\text{th}$ layer-plane of array C. Algorithm 4 shows the
pseudo-code used to multiply two N$\times$N$\times$N matrices.
\subsection{Code}
\begin{figure}[h!]
\includegraphics[width=\textwidth]{build/Algo4.png}
\end{figure}
%%%%%%%%%%%%%%%%%%%%%%%%%%%%%%%%%%%%%%%%%%%%%%%%%%%%%%%%%%%%%%%%%%%%%%%%%%%%%%%%
\section{CONCLUSIONS}
\cite{1}
 
%%%%%%%%%%%%%%%%%%%%%%%%%%%%%%%%%%%%%%%%%%%%%%%%%%%%%%%%%%%%%%%%%%%%%%%%%%%%%%%
%\clearpage %On a newpage
%\onecolumn
\bibliographystyle{build/witseie}
\bibliography{build/Ideal}

%{\tiny \vfill \hfill \today \hspace{5mm} witseie-paper-2003.\TeX}

\clearpage
\appendix

%%%%%%%%%%%%%%%%%%%%%%%%%%%%%%%%%%%%%%%%%%%%%%%%%%%%%%%%%%%%%%%%%%%%%%%%%%%%%%%%%%%%%%%%%%%%%%%%%%%%%%%%%%%%%%%%%%%%%%%%%%%%%%%%%%%%%%%%%%%
\end{document}

" vim: ts=4
" vim: tw=78
" vim: autoindent
" vim: shiftwidth=4
